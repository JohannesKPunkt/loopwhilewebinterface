\chapter{Docker container}
\label{chapter:docker}
This chapter describes how the loopwhile interactive interpreter service
is installed using a Docker container. The container bundles the application
together with an up-to-date nginx which is fully preconfigured.


\section{Installation}
The only requirement is an installation of Docker itself.
The application is installed in three steps:
\begin{itemize}
\item Clone the git repository of the interactive interpreter to the machine
      on which the service shall run
\item Run \verb|./build_container.sh| to build the container
\item Run \verb|./run_container.sh| to start the container. The service is then
      listening on port 80
\end{itemize}



\section{Logfiles}
In the directory that contains the \verb|./run_container.sh| script, a subdirectory
\verb|logs| will be created, which is used to store all relevant logfiles. There
you will find one subdirectory for each service:
\begin{itemize}
\item \verb|loopwhile|: Logfiles of the lw service
\item \verb|nginx|: Logfiles of nginx
\item \verb|supervisor|: Logfiles of the supervisord deamon. Additionally,
      stderr and stdout of all supervised services are captured.
\end{itemize}
